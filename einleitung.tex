\section{Einleitung}

\subsection{Prinzip der Laser-Triangulation}

\subsection{Herleitung}
Die zurückgelegte Entfernung ergibt sich aus der Differenz der Gesamtentfernung $z_{ges}$ des Startpunktes $z_0$.
\begin{equation}
	z = z_{ges} - z_0
	\label{eq:z1}
\end{equation}
Da sich die Strahlengänge in rechtwinklige Dreiecke unterteilen lassen, können durch Anwendung des Cosinus- und Sinussatzes Ausdrücke für $z_{ges}$ und $z_0$ bestimmt werden, die nur von der Entfernung $d$ zwischen Linse und Laser und den Winkeln $\alpha$ und $\beta$ abhängen.
\begin{align}
	\left.\begin{aligned}
		\cos(\beta-\alpha) &= \frac{z_{ges}}{g'}\\
		\sin(\beta-\alpha) &= \frac{d}{g'}
	\end{aligned}\qquad\right\}\qquad z_{ges} &= d\cdot\frac{cos(\beta-\alpha)}{\sin(\beta-\alpha)}\\[1em]
	\left.\begin{aligned}
		\cos(\beta) &= \frac{z_0}{g_1}\\ 
		\sin(\beta) &= \frac{d}{g_1}
	\end{aligned}\qquad\right\}\qquad z_0 &= d\cdot\frac{cos(\beta)}{\sin(\beta)}
\end{align}
Die entstandenen Terme können nun in Gleichung~(\ref{eq:z1}) eingesetzt werden. Wie man leicht sieht lassen sich die Brüche der Winkelfunktionen in den Cotangens überführen. 
\begin{align}
	z &= d\cdot\left(\frac{\cos(\beta-\alpha)}{\sin(\beta-\alpha)}-\frac{\cos(\beta)}{\sin(\alpha)}\right)\\
	  &= d\cdot\left(\cot(\beta-\alpha)-\cot(\beta)\right)
	  \label{eq:z2}
\end{align}
Da der Winkel $\alpha$ nicht direkt gemessen werden kann muss ein Ausdruck gefunden werden, der von den Parametern des Optischen Systems abhängt. Dazu werden zunächst die Punktverschiebung auf dem CCD-Sensor $\Delta p'$ und die Bildweite $b$ als bekannt angenommen.
\begin{equation}
	\left.\begin{aligned}
		\sin(\alpha) &= \frac{\Delta p'}{b'}\\
	\cos(\alpha) &= \frac{b}{b'}
	\end{aligned}\qquad\right\}\qquad
	\frac{\sin(\alpha)}{\cos(\alpha)} = \frac{\Delta p'}{b} = \tan(\alpha)
\end{equation}
Durch Umkehrung des Tangens erhält man nun für $\alpha$:
\begin{equation}
	\alpha = \arctan\left(\frac{\Delta p'}{b}\right)
	\label{eq:alpha}
\end{equation}
Da auch der Abstand $d$ zwischen Brennpunkt und Laser nicht genau bestimmt werden kann, ist es notwendig diesen durch bekannte Größen zu substituieren. Abhilfe schafft hier die Anwendung des Sinussatzes auf den Winkel $\beta$.
\begin{align}
	\sin(\beta) &= \frac{d}{g_1}\\
	d &= g_1\cdot\sin(\beta)
	\label{eq:d}
\end{align}
Setzt man nun (\ref{eq:alpha}) und (\ref{eq:d}) in Gleichung~(\ref{eq:z2}) ein, so ergibt sich folgender Ausdruck:
\begin{equation}
	z = g_1\cdot\sin(\beta)\cdot\left\{\cot\left[\beta-\arctan\left(\frac{\Delta p'}{b}\right)\right]-\cot(\beta)\right\}
	\label{eq:z3}
\end{equation}
Mit Hilfe der Linsengleichung (\ref{eq:linsengleichung}) muss nun noch ein passender Ausdruck für $b$ gefunden werden.
\begin{align}
	\frac{1}{f} &= \frac{1}{b} + \frac{1}{g} \label{eq:linsengleichung}\\
	b &= \frac{f\cdot g}{g-f} \label{eq:b}
\end{align}
Durch das Einsetzen von (\ref{eq:b}) in Gleichung~(\ref{eq:z3}) ergibt sich ein Ausdruck für $z$, der nur von bekannten oder messbaren Größen abhängt. 
\begin{equation}
	z = g_1\cdot\sin(\beta)\cdot\left\{\cot\left[\beta-\arctan\left(\frac{\Delta p'}{\frac{f\cdot g}{g-f}}\right)\right]-\cot(\beta)\right\}
	\label{eq:zfinal}
\end{equation}